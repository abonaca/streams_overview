% imports
\usepackage{xcolor}
\usepackage{hyperref}
\hypersetup{
    colorlinks=true,
    linkcolor=black,
    filecolor=black,
    urlcolor=navy,
    citecolor=black,
}
\urlstyle{same}

% Packages / projects / programming
\newcommand{\package}[1]{\texttt{#1}}
\newcommand{\acronym}[1]{{#1}}
\newcommand{\github}{\textsl{GitHub}}
\newcommand{\python}{\textsl{Python}}
\newcommand{\jax}{\package{JAX}}
\newcommand{\agama}{\package{Agama}}
\newcommand{\gala}{\package{gala}}

% Missions
\newcommand{\gaia}{\textsl{Gaia}}
\newcommand{\hipparcos}{\textsl{HIPPARCOS}}
\newcommand{\dr}[1]{\acronym{DR}#1}
\newcommand{\apogee}{\acronym{APOGEE}}
\newcommand{\sdss}{\acronym{SDSS}}
\newcommand{\sdssiv}{\acronym{SDSS-IV}}

% Stats / probability
\newcommand{\given}{\,|\,}
\newcommand{\norm}{\mathcal{N}}
\newcommand{\pdf}{\textsl{pdf}}

% Maths
\newcommand{\dd}{\mathrm{d}}
\newcommand{\deriv}[2]{\frac{\mathrm{d}{#1}}{\mathrm{d}{#2}}}
\newcommand{\dderiv}[2]{\frac{\mathrm{d^2}{#1}}{\mathrm{d}{#2}^2}}
\newcommand{\Deriv}[2]{\frac{\mathrm{D}{#1}}{\mathrm{D}{#2}}}
\newcommand{\pderiv}[2]{\frac{\partial {#1}}{\partial {#2}}}
\newcommand{\ppderiv}[2]{\frac{\partial^2 {#1}}{\partial {#2}^2}}
\newcommand{\transpose}[1]{{#1}^{\mathsf{T}}}
\newcommand{\inverse}[1]{{#1}^{-1}}
\newcommand{\argmin}{\operatornamewithlimits{argmin}}
\newcommand{\mean}[1]{\left< #1 \right>}

% Non-scalar variables
\renewcommand{\vec}[1]{\ensuremath{\bs{#1}}}
\newcommand{\mat}[1]{\ensuremath{\mathbf{#1}}}

% Units:
% Workaround for siunitx + AASTeX
% https://tex.stackexchange.com/questions/192610/use-emulateapj-aastex-with-siunitx
\usepackage{savesym}
\savesymbol{tablenum}
\usepackage{siunitx}
\ifdefined\unit\else
  \ifdefined\NewCommandCopy
    \NewCommandCopy\unit\si
  \else
    \NewDocumentCommand\unit{O{}m}{\si[#1]{#2}}
  \fi
\fi
\restoresymbol{SIX}{tablenum}
\DeclareSIUnit\year{yr}
\DeclareSIUnit\parsec{pc}
\DeclareSIUnit\mag{mag}
\DeclareSIUnit\Msun{M_\odot}
\DeclareSIUnit\msun{M_\odot}
\DeclareSIUnit\Rsun{R_\odot}
\DeclareSIUnit\mas{mas}
\newcommand{\mas}{\unit{\milli\arcsecond}}
\newcommand{\muas}{\unit{\micro\arcsecond}}
\newcommand{\kms}{\unit{\km\per\s}}
\newcommand{\masyr}{\unit{\mas\per\year}}
\newcommand{\kpc}{\unit{\kilo\parsec}}
\newcommand{\usurfdens}{\unit{\msun.\parsec^{-2}}}
\newcommand{\uvoldens}{\unit{\msun.\parsec^{-3}}}

\usepackage{longtable}
% \sisetup{
% %     output-exponent-marker = \text{e},   % Use 'e' for scientific notation
% %     table-format = 1.1e1,               % Adjust the number format (X.YeZ format)
% %     tight-spacing = true,                  % Reduce space between the number and exponent
%     table-number-alignment = center,     % Center-align numbers (you can also try 'right')
% }

% Misc. formatting
\newcommand{\bs}[1]{\boldsymbol{#1}}

% Astronomy
\newcommand{\abun}[2]{\ensuremath{{[\mathrm{#1}/\mathrm{#2}]}}}
\newcommand{\feh}{\abun{Fe}{H}}
\newcommand{\afe}{\abun{\alpha}{Fe}}
\newcommand{\mgfe}{\abun{Mg}{Fe}}
\newcommand{\logg}{\ensuremath{\log g}}
\newcommand{\Teff}{\ensuremath{T_{\textrm{eff}}}}
\newcommand{\vsini}{\ensuremath{v\,\sin i}}
\newcommand{\ion}[2]{#1\,\textsc{\romannumeral #2}}
\newcommand{\lcdm}{\ensuremath{\Lambda}CDM}


% Dynamics
\newcommand{\df}{\acronym{DF}}
\newcommand{\zmax}{\ensuremath{z_{\textrm{max}}}}
\newcommand{\rper}{\ensuremath{r_{\textrm{per}}}}
\newcommand{\rapo}{\ensuremath{r_{\textrm{apo}}}}

% TO DO
\newcommand{\todo}[1]{{\color{red} TODO: #1}}
\newcommand{\placeholder}[1]{{\color{purple} #1}}
\newcommand{\ab}[1]{{\color{teal} AB: #1}}
\newcommand{\changes}[1]{{\textbf{#1}}}

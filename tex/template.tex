%%
%% Copyright 2007-2020 Elsevier Ltd
%%
%% This file is part of the 'Elsarticle Bundle'.
%% ---------------------------------------------
%%
%% It may be distributed under the conditions of the LaTeX Project Public
%% License, either version 1.2 of this license or (at your option) any
%% later version.  The latest version of this license is in
%%    http://www.latex-project.org/lppl.txt
%% and version 1.2 or later is part of all distributions of LaTeX
%% version 1999/12/01 or later.
%%
%% The list of all files belonging to the 'Elsarticle Bundle' is
%% given in the file `manifest.txt'.
%%
%% Template article for Elsevier's document class `elsarticle'
%% with harvard style bibliographic references

\documentclass[final,5p,times,twocolumn,authoryear]{elsarticle}

%% Use the option review to obtain double line spacing
%% \documentclass[authoryear,preprint,review,12pt]{elsarticle}

%% Use the options 1p,twocolumn; 3p; 3p,twocolumn; 5p; or 5p,twocolumn
%% for a journal layout:
%% \documentclass[final,1p,times,authoryear]{elsarticle}
%% \documentclass[final,1p,times,twocolumn,authoryear]{elsarticle}
%% \documentclass[final,3p,times,authoryear]{elsarticle}
%% \documentclass[final,3p,times,twocolumn,authoryear]{elsarticle}
%% \documentclass[final,5p,times,authoryear]{elsarticle}
%% \documentclass[final,5p,times,twocolumn,authoryear]{elsarticle}

%% For including figures, graphicx.sty has been loaded in
%% elsarticle.cls. If you prefer to use the old commands
%% please give \usepackage{epsfig}

%% The amssymb package provides various useful mathematical symbols
\usepackage{amssymb}
%% The amsthm package provides extended theorem environments
%% \usepackage{amsthm}

%% The lineno packages adds line numbers. Start line numbering with
%% \begin{linenumbers}, end it with \end{linenumbers}. Or switch it on
%% for the whole article with \linenumbers.
%% \usepackage{lineno}

\journal{New Astronomy Reviews}

\begin{document}

\begin{frontmatter}

%% Title, authors and addresses

%% use the tnoteref command within \title for footnotes;
%% use the tnotetext command for theassociated footnote;
%% use the fnref command within \author or \affiliation for footnotes;
%% use the fntext command for theassociated footnote;
%% use the corref command within \author for corresponding author footnotes;
%% use the cortext command for theassociated footnote;
%% use the ead command for the email address,
%% and the form \ead[url] for the home page:
%% \title{Title\tnoteref{label1}}
%% \tnotetext[label1]{}
%% \author{Name\corref{cor1}\fnref{label2}}
%% \ead{email address}
%% \ead[url]{home page}
%% \fntext[label2]{}
%% \cortext[cor1]{}
%% \affiliation{organization={},
%%            addressline={},
%%            city={},
%%            postcode={},
%%            state={},
%%            country={}}
%% \fntext[label3]{}

\title{Stellar Streams in the Gaia Era}

%% use optional labels to link authors explicitly to addresses:
%% \author[label1,label2]{}
%% \affiliation[label1]{organization={},
%%             addressline={},
%%             city={},
%%             postcode={},
%%             state={},
%%             country={}}
%%
%% \affiliation[label2]{organization={},
%%             addressline={},
%%             city={},
%%             postcode={},
%%             state={},
%%             country={}}

\author[ociw]{Ana~Bonaca}
\author[cca]{Adrian~M.~Price-Whelan}

\affiliation[ociw]{organization={The Observatories of the Carnegie Institution for Science},
            addressline={813 Santa Barbara Street},
            city={Pasadena},
            postcode={91101},
            state={CA},
            country={USA}}

\affiliation[cca]{organization={Center for Computational Astrophysics},
            addressline={},
            city={New York},
            postcode={},
            state={NY},
            country={USA}}


\begin{abstract}
There are lots!
\end{abstract}

% %%Graphical abstract
% \begin{graphicalabstract}
% %\includegraphics{grabs}
% \end{graphicalabstract}
%
% %%Research highlights
% \begin{highlights}
% \item Research highlight 1
% \item Research highlight 2
% \end{highlights}

\begin{keyword}
%% keywords here, in the form: keyword \sep keyword

%% PACS codes here, in the form: \PACS code \sep code

%% MSC codes here, in the form: \MSC code \sep code
%% or \MSC[2008] code \sep code (2000 is the default)

\end{keyword}

\end{frontmatter}

%% \linenumbers

%% main text
\section{Introduction}
\label{sec:intro}
% AB
- Tidal debris
-- How they form, What are they?
-- Dwarf vs. globular cluster
-- Shell vs stream, radial vs tangential
-- Here focus on still coherent debris, not phase-mixed structures
- First discoveries (Sgr, Pal 5, Helmi - first orbital clustering, the slow trickle of photometric streams, …)
- Motivation for what they tell us based on theory / early 2000s work
- Fitting orbits to streams and then stream =/= orbit
- All the things that streams are sensitive to: subhalos, bar, spiral arms, GMCs, internal DF/dissolution

- Why were we at the standstill on many/all of these before Gaia?
- major shifts in the field of stellar streams post Gaia:
-- explosion in the number of streams discovered (\S~\ref{sec:discovery})

\section{Stream discoveries in the Gaia era}
\label{sec:discovery}
% APW
- Brief history of discovery methods in the pre-Gaia era (maybe move that to intro?)
- New methods with new data (combining photometry + kinematics, streamfinder)
- New discoveries and census as of 2023
- Also: follow-up now more efficient so the chemistries now becoming more readily available

\subsection{Key opportunities}
- The point that (all - few) streams have MSTO below Gaia faint limit and what that does for density characterization (or in “stream structure” below?)
- (reorder by complexity, maybe :)): Automated discovery with quantified selection function, discovery in “hard” parts of parameter space (in the disk plane or galactic center, not tangent to sky plane, etc…), comparison to expectations from cosmological simulations and GC formation models


\section{Stream orbital histories}
\label{sec:orbits}
% AB
- Still much unknown about most streams
- But Gaia gives us access to kinematics -> +MW model = orbits
- “Direct” and Hierarchical associations of streams (Orphan/Chenab, Atlas-aliqa uma vs. hierarchical groups)
- Finding GC progenitors

\subsection{Key opportunities}
- More comprehensive spectroscopic follow-up, potential modeling of a flexible potential, Stellar populations and element abundances, comparison to surviving dwarf galaxies and clusters (internal properties and orbits)


\section{Stream structure}
\label{sec:structure}
% APW
- Gaps and off-track features common at low surface brightness
- E.g., GD-1, Jhelum, AAU
- also motivated uncovering low-sb in other data sets (e.g., Pal 5, Jet)

\subsection{Key opportunities}
- density modeling robust to survey selection functions and dust


\section{Outlook}
\label{sec:outlook}
- Due to large number of streams: transitioning from stamp-collecting to population studies
- All of this will need improved theoretical modeling of the new Gaia discoveries (+ outlook for the future)
- Due to more secure membership probabilities: follow-up now more efficient so the chemistries now becoming more readily available


%% The Appendices part is started with the command \appendix;
%% appendix sections are then done as normal sections
%% \appendix

%% \section{}
%% \label{}


 \bibliographystyle{model2-names-astronomy}
 \bibliography{refs}

\end{document}

\endinput
%%
%% End of file `elsarticle-template-harv.tex'.
